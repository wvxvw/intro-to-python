% Created 2017-12-22 Fri 16:12
\documentclass[11pt]{article}
\usepackage[utf8]{inputenc}
\usepackage[T1]{fontenc}
\usepackage{fixltx2e}
\usepackage{graphicx}
\usepackage{longtable}
\usepackage{float}
\usepackage{wrapfig}
\usepackage{rotating}
\usepackage[normalem]{ulem}
\usepackage{amsmath}
\usepackage{textcomp}
\usepackage{marvosym}
\usepackage{wasysym}
\usepackage{amssymb}
\usepackage{hyperref}
\tolerance=1000
\usepackage[utf8]{inputenc}
\usepackage{commath}
\usepackage{pgf}
\usepackage{tikz}
\usetikzlibrary{shapes,backgrounds,arrows.meta}
\usetikzlibrary{arrows.meta}
\usetikzlibrary{positioning}
\usepackage{marginnote}
\usepackage{listings}
\usepackage{enumerate}
\usepackage{algpseudocode}
\usepackage{algorithm}
\usepackage{mathtools}
\setlength{\parskip}{16pt plus 2pt minus 2pt}
\renewcommand{\arraystretch}{1.6}
\author{Oleg Sivokon}
\date{\textit{<2017-12-01 Fri>}}
\title{Assignment 3, Flowcharts}
\hypersetup{
  pdfkeywords={Data-Structures, Algorithms, Assignment},
  pdfsubject={Third assignment in the course Intro to Python},
  pdfcreator={Emacs 25.3.1 (Org mode 8.2.10)}}
\begin{document}

\maketitle
\tableofcontents


\definecolor{codebg}{rgb}{0.96,0.99,0.8}
\definecolor{codestr}{rgb}{0.46,0.09,0.2}
\lstset{%
  backgroundcolor=\color{codebg},
  basicstyle=\ttfamily\scriptsize,
  breakatwhitespace=false,
  breaklines=false,
  captionpos=b,
  framexleftmargin=10pt,
  xleftmargin=10pt,
  framerule=0pt,
  frame=tb,
  keepspaces=true,
  keywordstyle=\color{blue},
  showspaces=false,
  showstringspaces=false,
  showtabs=false,
  stringstyle=\color{codestr},
  tabsize=2
}
\lstnewenvironment{maxima}{%
  \lstset{%
    backgroundcolor=\color{codebg},
    escapeinside={(*@}{@*)},
    aboveskip=20pt,
    captionpos=b,
    label=,
    caption=,
    showstringspaces=false,
    frame=single,
    framerule=0pt,
    basicstyle=\ttfamily\scriptsize,
    columns=fixed}}{}
}
\makeatletter
\newcommand{\verbatimfont}[1]{\renewcommand{\verbatim@font}{\ttfamily#1}}
\makeatother
\verbatimfont{\small}%
\clearpage

\section{Compare function}
\label{sec-1}
This function used to be in Python 2 standard library, but for no real reason
it was removed fro Python 3.  Here we give a possible implementation

\subsection{Algorithm}
\label{sec-1-1}
This algorithm relies on the fact that Python will convert logical values
into integers when they are used as arguments to arithmetic operations,
specifically, subtraction.  Thus:
\begin{enumerate}
\item \verb~True - False = 1~
\item \verb~True - True = 0~
\item \verb~False - False = 0~
\item \verb~False - True = -1~
\end{enumerate}

\begin{algorithm}
  \caption{Three-way comparison function a.k.a. arithmetic if (Fortran)
    \label{alg:cmp-function}}
  \begin{algorithmic}[1]
    \Require{$x$ and $y$ are drawn from a set with some (partial or total) ordering}
    \Statex
    \Function{Compare}{$x, y$}
    \State{$z \leftarrow x < y$} \Comment{$z$ is still a logical variable}
    \State{$q \leftarrow x > y$}
    \Comment{$q$ will only equal to $z$ if neither $x > y$ nor $x < y$}
    \State \Return{$z - q$}
    \EndFunction
  \end{algorithmic}
\end{algorithm}
\newpage

\subsection{Implementation}
\label{sec-1-2}
The implementation of the algorithm given above:

\lstset{language=Python,label= ,caption= ,numbers=none}
\begin{lstlisting}
def compare(a, b):
    '''
    Arguments ``a`` and ``b`` are drawn from the same ordered set.
    This function makes no assumptions about whether the order is
    total or partial.

    This function returns:
        1  if a < b
        -1 if a > b
        0  if a = b

    :param a: Must be comparable to ``b``.
    :param b: Must be comparable to ``a``.
    '''
    return ((a > b) - (a < b))
\end{lstlisting}
% Emacs 25.3.1 (Org mode 8.2.10)
\end{document}
