% Created 2017-12-29 Fri 18:32
\documentclass[11pt]{article}
\usepackage[utf8]{inputenc}
\usepackage[T1]{fontenc}
\usepackage{fixltx2e}
\usepackage{graphicx}
\usepackage{longtable}
\usepackage{float}
\usepackage{wrapfig}
\usepackage{rotating}
\usepackage[normalem]{ulem}
\usepackage{amsmath}
\usepackage{textcomp}
\usepackage{marvosym}
\usepackage{wasysym}
\usepackage{amssymb}
\usepackage{hyperref}
\tolerance=1000
\usepackage[utf8]{inputenc}
\usepackage{commath}
\usepackage{pgf}
\usepackage{tikz}
\usetikzlibrary{shapes,backgrounds,arrows.meta}
\usetikzlibrary{arrows.meta}
\usetikzlibrary{positioning}
\usepackage{marginnote}
\usepackage{listings}
\usepackage{enumerate}
\usepackage{algpseudocode}
\usepackage{algorithm}
\usepackage{mathtools}
\setlength{\parskip}{16pt plus 2pt minus 2pt}
\renewcommand{\arraystretch}{1.6}
\author{Oleg Sivokon}
\date{\textit{<2017-12-29 Fri>}}
\title{Assignment 7, Flowcharts}
\hypersetup{
  pdfkeywords={Data-Structures, Algorithms, Assignment},
  pdfsubject={Seventh assignment in the course Intro to Python},
  pdfcreator={Emacs 25.3.1 (Org mode 8.2.10)}}
\begin{document}

\maketitle
\tableofcontents


\definecolor{codebg}{rgb}{0.96,0.99,0.8}
\definecolor{codestr}{rgb}{0.46,0.09,0.2}
\lstset{%
  backgroundcolor=\color{codebg},
  basicstyle=\ttfamily\scriptsize,
  breakatwhitespace=false,
  breaklines=false,
  captionpos=b,
  framexleftmargin=10pt,
  xleftmargin=10pt,
  framerule=0pt,
  frame=tb,
  keepspaces=true,
  keywordstyle=\color{blue},
  showspaces=false,
  showstringspaces=false,
  showtabs=false,
  stringstyle=\color{codestr},
  tabsize=2
}
\lstnewenvironment{maxima}{%
  \lstset{%
    backgroundcolor=\color{codebg},
    escapeinside={(*@}{@*)},
    aboveskip=20pt,
    captionpos=b,
    label=,
    caption=,
    showstringspaces=false,
    frame=single,
    framerule=0pt,
    basicstyle=\ttfamily\scriptsize,
    columns=fixed}}{}
}
\makeatletter
\newcommand{\verbatimfont}[1]{\renewcommand{\verbatim@font}{\ttfamily#1}}
\makeatother
\verbatimfont{\small}%
\clearpage

\section{Sorting Lines Synchronously}
\label{sec-1}
This is an extremely simple task, since all the code was, essentially, written
before me.  All I need is to compose the following methods:
\verb~io.IOBase.readlines()~ to \verb~sorted()~ to \verb~io.IOBase.writelines()~ in this
exact order.  First will read lines, second will sort them and the third
will write them sorted.

Unfortunately, the example input file contains some junk (incorrect line
delimiters and empty lines).  Thus we also need to sanitize and filter
what we read from \verb~readlines()~.

\subsection{Algorithm}
\label{sec-1-1}
\begin{algorithm}
  \caption{Read and sort lines in a file
    \label{alg:read-sort-lines-function}}
  \begin{algorithmic}[1]
    \Require{$input$ exists and is readable}
    \Statex
    \Function{Sync-read-sort}{$input, output$}
    \State \Return{$write(sort(filter(read(file))), output)$}
    \EndFunction
  \end{algorithmic}
\end{algorithm}

\subsection{Implementation}
\label{sec-1-2}

\lstset{language=Python,label= ,caption= ,numbers=none}
\begin{lstlisting}
# Assuming `source' parameter is an already opened file,
# and the writing to file will be taken care of by another
# procedure:

def simple_sort_lines(source):
    return sorted(line for line in source if line.strip())
\end{lstlisting}

Some explanation may be due.

\begin{enumerate}
\item \verb~sorted()~ accepts iterators as well as lists, and in such case, will only
read the values once, so no extra reading will be performed.
\item \verb~sorted()~ produces a list.  It is possible, to write an algorithm which
generates sorted lines on-line, without having to deal with the entire
list at a time, but this would take too much space to write about here,
and the implementation would be quite complex.  However, I note that
\verb~nth_order_statistic()~ could be used for such task.
\item File-like objects in Python are also iterators, and it just so
conveniently happened that the iterator implemented by file-like objects
iterates by line.
\item The lines thus produced have \verb~\n~ character appended to them.  This is
intentional, since \verb~io.IOBase.writelines()~ doesn't append it on its own.
\end{enumerate}

\section{Sorting Lines Asynchronously}
\label{sec-2}
This is a more interesting task.  But the idea is still very simple:  I can
split the work between multiple workers, where each worker will sort its
chunk of the lines it reads, then I could collect all chunks and sort them
together.

The algorithm thus uses a popular in concurrent programming scheme:
``Scatter-Join''.

The non-trivial parts of this algorithm deal with ensuring that no worker
reads the input intended for another worker, and that merging doesn't do too
much work.  First is ensured by that each worker will skip ahead to the first
line break after seeking to an arbitrary position in file.  This is while each
worker will also read only full lines, while it is allowed to read more than
the chunk allocated to it.  This means that some parts of the file will be
read more than once, and that the worker who gets to read the very beginning
of the file is not allowed to skip the first line it reads.

\subsection{Algorithm}
\label{sec-2-1}
I will only illustrate the concerns raised in the previous chapter:

We can easily see that we can pre-compute all the necessary information to
make inner \verb~while~ loops independent, and so we can split the work between
independent workers.

\begin{algorithm}
  \caption{Ensure workload is distributed evenly and all lines are read
    \label{alg:workload-distribution}}
  \begin{algorithmic}[1]
    \Require{$length(f)$ gives the length of the file $f$}
    \Require{$seek(f, n)$ seeks to the position $n$ in file $f$}
    \Require{$position(f)$ gives the position at which $f$ is being read}
    \Require{$append(a, b)$ appends lists $a$ and $b$}
    \Require{$merge(lists)$ produces a sorted list with elements drawn from each list in $lists$}
    \Statex
    \Function{Distribute-work}{$input, workers$}
    \State{$size \gets length(input)$}
    \State{$lines \gets$ Empty List}
    \For{\texttt{$i \gets 0$; $i < workers$; $i \gets i + 1$}}
    \State{$seek(input, i * \frac{size}{workers})$}
    \State{$lines_i \gets$ Empty List}
    \If{$i \neq 0$}
    \Comment{Skip line that will be read by the previous worker}
    \State{$readline(input)$}
    \EndIf
    \While{$position(file) < (i + 1) * \frac{size}{workers}$}
    \State{$lines_i \gets append(lines_i, readline(input))$}
    \EndWhile
    \State{$lines \gets append(lines, lines_i)$}
    \EndFor
    \State \Return{$merge(lines)$}
    \EndFunction
  \end{algorithmic}
\end{algorithm}

\begin{algorithm}
  \caption{Merges sorted arrays $A$ and $B$ to obtain third sorted array $C$
    \label{alg:merge}}
  \begin{algorithmic}[1]
    \Require{$length(X)$ gives the length of array $X$}
    \Statex
    \Function{Merge}{$A, B$}
    \State{$alength \gets length(A)$}
    \State{$blength \gets length(B)$}
    \State{$apos \gets 0$}
    \State{$bpos \gets 0$}
    \State{$clength \gets alength + blength$}
    \State{$C \gets array(clength)$}
    \State{$cpos \gets 0$}
    \While{$alength > apos \land blength > bpos$}
    \If{$A_{apos} \leq B_{bpos}$}
    \State{$C_{cpos} \gets A_{apos}$}
    \State{$apos \gets apos + 1$}
    \Else
    \State{$C_{cpos} \gets B_{bpos}$}
    \State{$bpos \gets bpos + 1$}
    \EndIf
    \State{$cpos \gets cpos + 1$}
    \EndWhile
    \If{$apos = alength$}
    \State{$remainder \gets B$}
    \State{$rpos \gets bpos$}
    \Else
    \State{$remainder \gets A$}
    \State{$rpos \gets apos$}
    \EndIf
    \While{$rpos < length(remainder)$}
    \State{$C_{cpos} \gets remainder_{rpos}$}
    \State{$cpos \gets cpos + 1$}
    \State{$rpos \gets rpos + 1$}
    \EndWhile
    \State \Return{$C$}
    \EndFunction
  \end{algorithmic}
\end{algorithm}
\pagebreak

The \verb~merge()~ function is the same one used in textbook \verb~merge-sort()~
procedure.  For completeness, the pseudocode is given below:

\subsection{Implementation}
\label{sec-2-2}
The implementation of the algorithm given above:

\lstset{language=Python,label= ,caption= ,numbers=none}
\begin{lstlisting}
def merge_sync(left, right):
    left_pos, right_pos = 0, 0
    left_size, right_size = len(left), len(right)
    result, remainder = [None] * (left_size + right_size), None
    result_pos, remainder_pos = 0, 0

    while left_pos < left_size and right_pos < right_size:
        if left[left_pos] <= right[right_pos]:
            result[result_pos] = left[left_pos]
            left_pos += 1
        else:
            result[result_pos] = right[right_pos]
            right_pos += 1
        result_pos += 1

    if left_pos < left_size:
        remainder = left
        remainder_pos = left_pos
    else:
        remainder = right
        remainder_pos = right_pos

    while result_pos < len(result):
        result[result_pos] = remainder[remainder_pos]
        result_pos += 1
        remainder_pos += 1
    return result


async def sort_chunk(queue, source, start, end):
    chunk = []
    source.seek(start)
    source.readline()
    while source.tell() < end:
        line = source.readline().strip()
        if line:
            chunk.append(line + '\n')
    await queue.put(sorted(chunk))


async def merge(queue, expected_count):
    result = []
    for _ in range(expected_count):
        result = merge_sync(result, await queue.get())
    return result


def async_merge_sort_lines(source, fsize, coroutines_count=0):
    coroutines_count = coroutines_count or multiprocessing.cpu_count() - 1

    with closing(asyncio.new_event_loop()) as loop:
        queue = asyncio.Queue(loop=loop)
        chunk_size = (fsize // coroutines_count) + 1
        readers = [
            sort_chunk(queue, source, x, min(x + chunk_size, fsize))
            for x in range(0, fsize, chunk_size)
        ]
        writer = merge(queue, coroutines_count)
        future = asyncio.gather(*(readers + [writer]), loop=loop)
        loop.run_until_complete(future)
        return future.result()[-1]
\end{lstlisting}
% Emacs 25.3.1 (Org mode 8.2.10)
\end{document}
