% Created 2017-12-01 Fri 13:43
\documentclass[11pt]{article}
\usepackage[utf8]{inputenc}
\usepackage[T1]{fontenc}
\usepackage{fixltx2e}
\usepackage{graphicx}
\usepackage{longtable}
\usepackage{float}
\usepackage{wrapfig}
\usepackage{rotating}
\usepackage[normalem]{ulem}
\usepackage{amsmath}
\usepackage{textcomp}
\usepackage{marvosym}
\usepackage{wasysym}
\usepackage{amssymb}
\usepackage{hyperref}
\tolerance=1000
\usepackage[utf8]{inputenc}
\usepackage{commath}
\usepackage{pgf}
\usepackage{tikz}
\usetikzlibrary{shapes,backgrounds,arrows.meta}
\usetikzlibrary{arrows.meta}
\usetikzlibrary{positioning}
\usepackage{marginnote}
\usepackage{listings}
\usepackage{enumerate}
\usepackage{algpseudocode}
\usepackage{algorithm}
\usepackage{mathtools}
\setlength{\parskip}{16pt plus 2pt minus 2pt}
\renewcommand{\arraystretch}{1.6}
\author{Oleg Sivokon}
\date{\textit{<2017-12-01 Fri>}}
\title{Assignment 3, Flowcharts}
\hypersetup{
  pdfkeywords={Data-Structures, Algorithms, Assignment},
  pdfsubject={Third assignment in the course Intro to Python},
  pdfcreator={Emacs 25.3.1 (Org mode 8.2.10)}}
\begin{document}

\maketitle
\tableofcontents


\definecolor{codebg}{rgb}{0.96,0.99,0.8}
\definecolor{codestr}{rgb}{0.46,0.09,0.2}
\lstset{%
  backgroundcolor=\color{codebg},
  basicstyle=\ttfamily\scriptsize,
  breakatwhitespace=false,
  breaklines=false,
  captionpos=b,
  framexleftmargin=10pt,
  xleftmargin=10pt,
  framerule=0pt,
  frame=tb,
  keepspaces=true,
  keywordstyle=\color{blue},
  showspaces=false,
  showstringspaces=false,
  showtabs=false,
  stringstyle=\color{codestr},
  tabsize=2
}
\lstnewenvironment{maxima}{%
  \lstset{%
    backgroundcolor=\color{codebg},
    escapeinside={(*@}{@*)},
    aboveskip=20pt,
    captionpos=b,
    label=,
    caption=,
    showstringspaces=false,
    frame=single,
    framerule=0pt,
    basicstyle=\ttfamily\scriptsize,
    columns=fixed}}{}
}
\makeatletter
\newcommand{\verbatimfont}[1]{\renewcommand{\verbatim@font}{\ttfamily#1}}
\makeatother
\verbatimfont{\small}%
\clearpage

\section{Very simple calculator}
\label{sec-1}
This calculator works similar to earlieast \href{https://en.wikipedia.org/wiki/Hewlett-Packard_9100A}{HP 9100A calculator},
However, it can only add, subtract, divide and multiply.

\subsection{Reverse Polish Notation}
\label{sec-1-1}
Calculator uses reverse Polish notation (assignment doesn't specify
the notation to be used) because it is easier to implement, it
allows variable number of operands (assignment requires two
operands, but doesn't state bounds), and because this is typical in
professional calculators.

Conceptually, this means that operands are written before the
operation, the operation ``consumes'' all operands to the left of
it.  See the grammar below for more formal description:

\begin{align*}
  &S \to Num (Delim \; Num)^* Op \\
  &Delim \to \;\sim (Digit \;|\; Op)^+ \\
  &Op \to + \;|\; - \;|\; * \;|\; / \\
  &Digit \to [0..9] \\
  &Num \to Digit^+ \;.
\end{align*}


\subsection{Two stacks and two streams}
\label{sec-1-2}
Conceptually, the calculator operates on two stacks and two streams:

\begin{description}
\item[{character stack}] Is used to accumulate characters to later
interpret them as a number.
\item[{number stack}] Is used to accumulate numbers to later feed them
to the selected operation.

\item[{input stream}] Is used to read the input from user.
\item[{output stream}] Is used to print the results.
\end{description}

\subsection{Limitations}
\label{sec-1-3}
\begin{itemize}
\item To simplify the code, the calculator only works with positive
integers.
\item The limits on the number of arguments, size of integers etc. are
all the same as for the Python interpreter used.
\item Division by zero is handled by Python itself, not this program.
\end{itemize}

\section{Flowchart diagram}
\label{sec-2}

\tikzstyle{startstop} = [rectangle, rounded corners, minimum
  width=3cm, minimum height=1cm,text centered, draw=black,
  fill=red!30]

\tikzstyle{io} = [trapezium, trapezium left angle=70, trapezium right
  angle=110, minimum width=3cm, minimum height=1cm, text centered,
  draw=black, fill=blue!30]

\tikzstyle{process} = [rectangle, minimum width=3cm, minimum
  height=1cm, text centered, draw=black, fill=orange!30]

\tikzstyle{decision} = [diamond, minimum width=3cm, minimum
  height=1cm, text centered, draw=black, fill=green!30]

\tikzstyle{arrow} = [thick,->,>=Stealth]

\begin{tikzpicture}[align=center, node distance=2.5cm and 2cm]

  \node (start) [startstop] {Start};
  \node (in1) [io, below of=start] {Input character};
  \node (dec1) [decision, below of=in1] {Is delimiter?};
  \node (proc1) [process, right= of dec1] {Empty\\character\\stack};
  \node (proc2) [process, left= of dec1] {Push to\\character\\stack};
  \node (dec2) [decision, below of=proc1] {Is operator?};
  \node (proc3) [process, below of=dec2] {Empty\\number\\stack};
  \node (proc4) [process, below of=proc3] {Apply\\operator};
  \node (proc5) [process, below of=proc4] {Print\\result};
  \node (proc6) [process, left= of dec2] {Push to\\number\\stack};
  \node (dec3) [decision, left= of proc3] {Is number?};
  \node (proc7) [process, left= of dec3] {Display error};

  \draw [arrow] (start) -- (in1);
  \draw [arrow] (in1) -- (dec1);
  \draw [arrow] (dec1) -- node[anchor=south] {yes} (proc1);
  \draw [arrow] (dec1) -- node[anchor=south] {no} (proc2);
  \draw [arrow] (proc1) -- (dec2);
  \draw [arrow] (dec2) -- node[anchor=west] {yes} (proc3);
  \draw [arrow] (dec2) -- node[anchor=south] {no} (dec3);
  \draw [arrow] (dec3) -- node[anchor=west] {yes} (proc6);
  \draw [arrow] (dec3) -- node[anchor=south] {no} (proc7);
  \draw [arrow] (proc3) -- (proc4);
  \draw [arrow] (proc4) -- (proc5);

\end{tikzpicture}

To simplify the diagram, all transitions from final states to the input state
have been omitted.
% Emacs 25.3.1 (Org mode 8.2.10)
\end{document}
